\documentclass[10pt, a4paper]{article}
\usepackage[spanish]{babel}
\usepackage[utf8]{inputenc}
\usepackage[right=2cm,left=3cm,top=2cm,bottom=2cm,headsep=0cm,footskip=0.5cm]{geometry}
\usepackage{graphicx}
\usepackage{listings}
\usepackage{xcolor}

\definecolor{light-gray}{gray}{0.95}

\lstset{
	language = bash,
	basicstyle = \small\ttfamily,
	columns = flexible,
	breaklines = true,
	backgroundcolor = \color{light-gray},
	xleftmargin=0.5cm,
	frame=tlbr,
	framesep=4pt,
	framerule=0pt,
}


\linespread{1}
\setlength{\parskip}{1\baselineskip}
\parindent 1cm
\sloppy

%-------------------------------------------------------------------------

\begin{document}

%-------------------------------------------------------------------------
\thispagestyle{empty}
\begin{center}
\Large
\textbf{
UNIVERSIDAD NACIONAL DE SAN AGUSTIN\\
FACULTAD DE PRODUCCIÓN Y SERVICIOS\\
ESCUELA DE CIENCIA DE LA COMPUTACIÓN\\}

\vspace{1.5cm}
\begin{figure}[hb]
	\centering
	\includegraphics[width=0.6\textwidth]{jpg/logoCS_UNSA.jpg}
\end{figure}

\textbf{BASES DE DATOS I\\}
Prof a Cargo: Ing. Fredy Gonzales Saji\\
\vspace{1cm}
\textbf{PRACTICA Nro 1: INSTALACIÓN POSTGRESQL}

\vspace{2cm}

AMABLE ROMERO, DIEGO JAVIER\\
LLERENA HERMOZA, JOSE ADRIAN\\
PORTOCARRERO ESPIRILLA, MAX WILLIANS\\
TURPO APAZA, CRHISTIAN ANDREW\\
CERVANTES ALARCÓN, JHONATAN MARTI\\

\vfill

2016
\end{center}

%-------------------------------------------------------------------------
\newpage
\thispagestyle{empty}

\section{Resumen}
\paragraph{}
adfasdfasd lorem
\paragraph{}
asdfadsfa

\section{Bases de Datos}
\subsection{Definición}
aqui va la definición
\section{PostgreSQL}
\subsection{Definición}
\subsection{Historia}
\subsection{Características}

\section{Installación}
la versión que instalaremos será PostgreSQL 9.5
\subsection{CentOS 7 - Servidor}
Lo primero que vamos a hacer es agregar la dirección de la que vamos a descargar el PostgreSQL.

\begin{lstlisting}
rpm -Uvh https://download.postgresql.org/pub/repos/yum/9.5/redhat/rhel-7-x86_64/pgdg-centos95-9.5-2.noarch.rpm
\end{lstlisting}

Actualizamos nuestra lista de repositorios

\begin{lstlisting}
yum update
\end{lstlisting}

seguimos con la instalación del servidor y herramientas de desarrollo
\begin{lstlisting}
yum install postgresql95-server postgresql95-contrib
\end{lstlisting}


\subsection{Windows - Cliente}

\section{Conclusiones}
\begin{itemize}
\item Primera Conclusión
\item Primera Conclusión
\item Primera Conclusión
\end{itemize}

\end{document}
